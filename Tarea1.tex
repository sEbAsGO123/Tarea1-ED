\documentclass{article}
\usepackage[utf8]{inputenc}
\usepackage{amsmath} 
\usepackage{amssymb}

\title{Tarea1 ED}
\author{Arrieta Mancera Luis Sebastian\\ Pintor Muñoz Pedro Joshue\\ Gómez Rodriguez José Antonio }
\date{October 2020}

\begin{document}

\maketitle

\textbf{\Large Pregunta 1}\\
\textbf{\Large Pregunta 2}\\
\textbf{\Large Pregunta 3}\\
\textbf{\Large Pregunta 4}\\

 \textbf{Traduce formalmente a la lógica de proposiciones los siguientes enunciados. Indica el valor de cada variable atómica}.\\
    
a) Seras feliz solo si buscas el placer y no te dejas esclavizar por los deseos.
\begin{align*}
    &s=\text{serás feliz }\\
    &p=\text{buscas el placer}\\
    &e=\text{te dejas esclavizar por los deseos}
\end{align*}

b) Maŕıa fue al teatro el lunes en la noches solo en el caso de que no tuviera clase el martes temprano.

\begin{align*}
    &m=\text{María fue al teatro}\\
    &e=\text{tuviera clases el martes temprano}
\end{align*}

c) Si x es mayor que 3 entonces tambíen es mayor que y\\
\begin{align*}
    &x=\text{x mayor que 3}\\
    &y=\text{x mayor que y}
\end{align*}

d)Si los tres lados de un triangulo son congruentes, entonces los tres  ́angulos del triangulo son congruentes
\begin{align*}
    &t=\text{los tres lados de un triángulo son congruentes}\\
    &a=\text{los tres ángulos de un tríangulo son congruentes}
\end{align*}

\textbf{\Large Pregunta 5}\\
\textbf{Formaliza los siguientes argumentos lógicos. Indica el valor de cada variable atómica utilizada.}\\

a)Que un animal tenga espina dorsal es condición necesaria para que sea un animal vertebrado. Por otro
lado, si el animal tiene encéfalo y médula espinal, entonces tiene que ser vertebrado. Sabemos también,
que si un animal es vertebrado, entonces tiene cráneo. Por lo tanto podemos decir que todos los animales
que tiene cráneo, entonces tienen médula espinal.\\
\begin{align*}
    &d=\text{un animal tenga espina dorsal}\\
    &a=\text{sea un animal vertebrádo}\\
    &e=\text{el animal tiene encéfalo}\\
    &m=\text{el animal tiene médula espinal}\\
    &c=\text{el animal tiene cráneo}\\
\end{align*}
\begin{align*}
    &(a \rightarrow d)\land (e\land m \rightarrow a)\\
    &a\rightarrow c\\
    &\rule{30mm}{0.1mm}\\
    &\therefore c\rightarrow m\\
\end{align*}
b)Todo numero entero o es primo o es compuesto. Si es compuesto, es un producto de factores primos y si
es un producto de factores primos, entonces es divisible por ellos. Pero si un numero entero es primo, no
es compuesto, aunque es divisible por si mismos y por la unidad y consiguientemente, también divisible
por números primos. Por lo tanto, todo número entero es divisible por números primos.\\
\begin{align*}
    &e=\text{todo número entero}\\
    &p=\text{es primo}\\
    &c=\text{es compuesto}\\
    &f=\text{es un producto de factores primos}\\
    &d=\text{es divisible por factores primos}\\
    &d'=\text{es divisible por si mismo}\\
    &d''=\text{es divisible por la unidad}\\
\end{align*}
\begin{align*}
    &e\rightarrow p \lor c\\
    &(c\rightarrow f)\land (f\rightarrow d)\\
    &(e\rightarrow p)\rightarrow \neg c\\
    &d'\land d'' \land d\\
    &\rule{30mm}{0.1mm}\\
    &\therefore e\rightarrow d
\end{align*}



\textbf{\Large Pregunta 6}\\
\textbf{\Large Pregunta 7}\\
\textbf{\Large Pregunta 8}\\
\textbf{\Large Pregunta 9}\\
\textbf{\Large Pregunta 10}\\
\textbf{\Large Pregunta 11}\\





\end{document}
